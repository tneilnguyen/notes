\documentclass{article}
\usepackage{geometry}
\frenchspacing
\begin{document}
\title{Classics 17B: From Marcellus to Caesar}
\author{Scribe: Tyler Nguyen}
\date{Lecture 8: February 14, 2017}
\maketitle
\section{Punic Wars to the end of the Republic}
\subsection{Punic Wars to the end of 1st Century BCE}
\begin{itemize}
\item 218--201 BCE Hannibal invades Italy in 2nd Punic War
\item 211 BCE Marcellus conquers Syracuse (Sicily) and displays Greek art in his triumph
\item 197 BCE Flaminius liberates Greece from Macedonian rule
\item 179 BCE Basilica Aemilia, Roman Forum
\item 146 BCE End of Punic Wars; Greece made a Roman province
\item 100 BCE Julius Caesar born
\item 91--88 BCE Social War
\item 61 BCE Pompey awarded triumph for his African victory
\item 55 BCE Theater of Pompey built in Rome
\item 54 BCE Forum Iulium built in Rome
\item 49 BCE Caesar crosses the Rubicon River (border in N. Italy) civil wars break out
\item 48 BCE Caesar defeats Pompey at Pharsalus
\item 47 BCE Caesar in Egypt---makes political and personal alignment with Cleopatra
\item 44 BCE Caesar issues coin with his portrait; Caesar assassinated in Rome
\end{itemize}
\subsection{Roman Generals and Greek Art}
The conquest of Syracuse in 211 BCE by Marcus Marcellus began the influx of Greek art into Rome and thus the long standing love affair that Romans had for anything Greek.  Marcellus is quoted as having said that his motive for bringing Greek art to Rome was twofold: it would make a visual impression on his triumph and beautify the city.

Roman Generals left their mark on the Greek landscape---Titus Quinctius Flaminius freed the old city-states in Greece from Macedonian rule in 197 BCE (then he made coins with his own image on them in Greece---something he could not have done in Rome!).  Aemilius Paullus defeats Perseus of Macedonia at Pydna in 168 BCE.  He builds the \textsc{Victory Monument of Aemilius Paullus, Delphi, Greece, 168 BCE}.  Plutarch reported that the spoils of war were so extensive that the triumph was spread over 3 days.
\section{Greek Art 101}
Greek statues in Rome: copies were made all over Italy of Greek masterpieces.  Some artists made hybrid statues, namely famous bits and pieces of several sculptures to make one.  Pasiteles from southern Italy is one such artist who worked in Rome in the 1st century BCE.

The \textsc{Paris-Munich Reliefs (Altar of Domitius Ahenobarbus), Rome, 2nd--1st BCE} show the following: census; sacrifice to Mars and enrollment of troops (very Roman); wedding of Neptune and Amphitrite (very Greek).
\section{Roman Ancestor cult, portraits}
Example: \textsc{Man with portrait busts of his ancestors, Rome, late 1st century BCE (marble)}.  Veristic style: super realistic portraits of men (for the most part).  \textsc{Head of an old man, Osimo, marble, 1st century BCE} has verism, receding hairline, wrinkles, and sunken eyes.  \textsc{Head of an old woman, Palombara Sabina, 40--30 BCE} has a roll of hair over the forehead, sunken cheeks, and a mole.  \textsc{Pseudo-Athlete, Delos, Greece, 1st century BCE} has a head in the style of Republican Rome, but the  body is youthful, heroic and idealized---Greek beauty.  The statue was found in a Roman general's home in Delos.  The \textsc{General from Sanctuary of Hercules, Tivoli, 75--50 BCE} also has a veristic head and idealized Greek body.

\section{Pompey and Caesar}
Pompey was a Roman General, Consul, and ally and son-in-law of Julius Caesar until 54 BCE when Julia died in childbirth.  He became Caesar's enemy and fled Rome when Caesar crossed the Rubicon in 49 BCE.  He died in Egypt in 48 BCE (beheaded).  Pompey's Theater, Rome's first permanent theater, was completed in 55 BCE and was the site of Caesar's assassination in 44 BCE.
\end{document}
