\documentclass{article}
\usepackage{geometry}
\frenchspacing
\begin{document}
\title{Classics 17B: Republican Town Planning and Pompeii}
\author{Scribe: Tyler Nguyen}
\date{Lecture 6: February 2, 2017}
\maketitle
\section{City Structure}
Romans adopted the grid plan of Greek and Etruscan cities; cites like Marzabotto and Ostia are evidence of this structure.  Roman armies also camped in grids after marching; soldiers were assigned blocks to be quartered in.  Often, city centers would be built at the intersection of the \textit{cardo} and \textit{decumanos}, streets built on north-south and east-west axes, respectively.
\section{Pompeii}
Pompeii contains an amphitheatre, a distinctly Roman structure, where citizens could enjoy bloodsport; Greeks built theatres.
\begin{itemize}
\item How has the rediscovery of Pompeii impacted Wester art, architecture, and culture?
\item How can ruins connect the past with the present or future?
\item Why are sites like Pompeii and Herculaneum so popular for tourists today?
\end{itemize}
Pliny the Younger describes the ``phenomenon'' of Mount Vesuvius's 79 CE eruption while Pliny the Elder sails out to investigate, dying as a result.
\subsection{The Exact Date of the Eruption}
August 24, 79 CE on Vulcanalia, the festival in honor of the god Vulcan.  There are arguments for both summer and autumn dates as well.
\subsection{The Archaeological record of Pompeii up until 79 CE}
\begin{itemize}
\item 10th--8th: Early settlement at Pompeii
\item 8th--6th: Greeks and Etruscans in Campania
\item 6th: Streets established, Temple of Apollo and the Doric Temple
\item 5th: Samnites take control of Campania
\item 4th: Pompeii is a Samnite city
\item 4th/3rd: Rome conquers Bay of Naples
\item 218--210 BCE: Economic boom in Pompeii
\item 91--88 BCE: Social War between Roman and allied cities
\item 80 BCE: Sulla converts Pompeii into a Roman colony, Romans move in
\item 62 CE: Earthquake at Pompeii---extensive damage
\item 79 CE: Vesuvius erupts
\end{itemize}
\subsection{Pompeii After the Eruption}
\begin{itemize}
\item Help from Rome---Emperor Titus, aid going to neighboring towns not wiped out from eruption.
\item Clear evidence of salvaging or plunder from immediately after the eruption to the present.
\item All statues in the Forum are missing, as well as marble veneers.
\item Where were the cult statues (*Capitolium fragments, Aesculapius)?
\item Holes in houses by tunnellers in search of relics or those trying to escape.
\item ``Digging'' started first at Herculaneum, then shifted to Pompeii.
\item Kingdom of Naples (separate state, Italy was not unified until 1871) passed into the hands of foreign rulers, the Bourbons, the French, the Bourbons again, etc.
\item The first buildings to come to 'light' at Pompeii were in the theater district: the Temple of Isis (1764), the large theater (1765), the Triangular Forum (1765).  Bodies began to be unearthed too and became a morbid attraction for curious visitors to the site.  The motive for digging was to find buried treasures.
\item Giuseppe Fiorelli (1823--1896 once imprisoned by the Bourbons) was responsible for: stopping the removal of paintings; starting the antiquarium; creating the address system and plaster casts and much more.
\end{itemize}
\end{document}
