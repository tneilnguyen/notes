\documentclass{article}
\usepackage{geometry}
\frenchspacing
\begin{document}
\title{Classics 17B: From Village to World Capital}
\author{Scribe: Tyler Nguyen}
\date{Lecture 4: January 26, 2017}
\maketitle
Temples permitted only priests and other religious officials to enter.  Worshippers could only leave votives outside the temple.  Etruscan temples had terra cotta tiles and wooden support beams and columns.
\section{The Tarquins in Rome}
\begin{itemize}
\item Rome's first great building program began with the Etruscan kings.
\item Etruscan artists, architects, merchants, and more flocked to Rome.
\item When the Monarchy fell, Rome still looked like an Etruscan city.
\item The Roman Republic established: consuls, enlarged the Senate and created various magistrates in addition to the \textit{Cursus Honorum}.
\end{itemize}
According to legend, Rome's transition from Monarchy to Republic was in part triggered by Tarquinius Sextus's rape of Lucretia, wife of a Roman.
\section{The Roman Republic and walls}
\begin{itemize}
\item \textit{Servian} Wall (4th century BCE)
\item Aurelian Wall
\end{itemize}
\section{Fate of the Etruscans}
All of Etruria fell to the Romans by the 2nd century BCE.  Romans sacked the cities and required all official correspondence and transations to be conducted in Latin.
\section{Republican Buildings}
Temples like the \textsc{Republican Temples, Largo Argentina, ca. 4th to 1st century BCE} were built to celebrate military victories.  \textsc{Pompey's Theater, ca. 55 BCE} is a theater built behind the four Republican temples, with a temple to Venus attached to justify its construction.

The \textsc{Temple of Vesta or Hercules Victor, ca. mid 2nd century BCE} was a round temple on the banks of the Tiber.  The \textsc{Sanctuary of Fortuna, Palestrine (Praeneste), late 2nd century BCE} is one of the first examples of Roman cement construction.  It was a seven-floor sanctuary built into a hillside and contained barrel-vaulted ceilings made possible by cement.

\textit{Opus incertum} is a building technique that involves using cement to join small rocks, often used to build walls.  The \textsc{Temple of Portunus, ca. 75 BCE, Rome} was dedicated to Portunus, god of keys, doors, and livestock.  The temple was \textit{pseudoperipteral} and combined Greek and Etruscan influence; the columns had Ionic capitals, which succeeded Doric capitals.  Later, Corinthian capitals, decorated with acanthus leaves, became more prevalent.

The \textsc{Sanctuary of Jupiter Anxur, Terracina, ca. 80 BCE} also contained barrel-vaulted ceilings and was built on the shore so that it would be visible to passing merchant ships and project Roman power.

\textit{Opus reticulatum} used diamond-shaped bricks around cement with a net-like pattern of mortar for structural integrity.
\section{Roman Gains during the Republic}
\begin{itemize}
\item Territory
\item Wealth
\item Art
\item Military strength
\item Use of cement
\item Latin as official language
\end{itemize}
\end{document}
