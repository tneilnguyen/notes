\documentclass{article}
\usepackage{geometry}
\frenchspacing
\begin{document}
\title{Classics 17B: Preparing for the Afterlife during the Early Empire}
\author{Scribe: Tyler Nguyen}
\date{Lecture 12: February 28, 2017}
\maketitle
\section{Notes on burial in ancient Rome}
\begin{itemize}
\item Both cremation and inhumation were used in ancient Roman times...
\item Tombs were placed outside the city walls on the very roads that lead in and out of the city.
\item Streets lined with tombs became very popular sites for graffiti, advertising, etc.  Who is important in town?
\item Where are the children and the slaves?
\item Tomb structures range from very simple to elaborate (depended on your pocketbook).
\item Religious rites would have been performed at the gravesite.
\item Most of the tombs contain multiple burials with ashes placed inside the tomb in urns.
\item Offerings to keep the souls alive ... in the form of food, wine, oil, blood, etc.  Funerary meals at the tomb!
\item Corpse had to be cleaned, purified, etc.
\item Funerals were often organized by professional undertakers---there were funeral clubs too (to cover expenses).
\item Tombs and tombstones were the best insurance policy against oblivion.
\item Coin in the mouth for Charon, the ferryman who takes you across the river Styx in the underworld.
\end{itemize}
\end{document}
