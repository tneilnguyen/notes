\documentclass{article}
\usepackage{geometry}
\frenchspacing
\begin{document}
\title{Classics 17B: Introduction to Roman Archaeology}
\author{Scribe: Tyler Nguyen}
\date{Lecture 2: January 19, 2017}
\maketitle
\section{Italy Before the Rise of Rome}
The Etruscans ruled Northern Italy, while the Greeks settled Magna Graecia, Southern Italy.
\subsection{Architecture and Art}
Greek architects concerned with ``proportion and the cosmic order.'' Example: \textsc{Temple of Hera II or Apollo, Paestum, ca. 460 BCE}.  On public buildings, metopes were used to present a narrative to citizens, regardless of literacy. The only surviving example of Ancient Greek tomb painting is the \textsc{Tomb of the Diver, Paestum, ca. 470 BCE}.  It depicts a symposium, a male drinking party. It may be derivative of Etruscan painting.  The ceiling depicts a diver in an abstract landscape, contrasting with the more natural landscape of the Etruscan \textsc{Tomb of Hunting and Fishing, Tarqunia, ca. 520 BCE}.
\subsection{Etruscan Civilization}
Etruscan Villages:\\
\begin{tabular}{ll}
Caisra & Cerveteri\\
Tarchna & Tarquinia\\
Velch & Vulci\\
Veii & Veii\\
Vetluna & Vetulonia\\
Velzna & Orvieto\\
Clevsin & Chiusi\\
Perusia & Perugia\\
Curtun & Cortona\\
Fulfluna & Populonia\\
\end{tabular}

Etruscans built villages on hilltops for natural defense. Villanovans predated the Etruscans and practiced cremation. The gender of the deceased could often be discerned from items interred. Upper-class Etruscans began to practice inhumation. Example: \textsc{Regolini-Galassi Tomb, ca. 7th century BCE}. Later apartment-like tumuli sold plots for families. Exmaple: \textsc{Banditaccia Necropolis, Cerveteri}. Sarcophagi were made of dense clay, often with a layer of terra cotta. Example: \textsc{Sarcophagus of the ``married couple,'' Cerveteri, ca. 520 BCE}.  Etruscans provides us with the earliest and largest body of examples of pre-Roman wall painting. Example: \textsc{Tomb of the Bulls, Tarquinia, ca. 540-530 BCE}. Etruscans adopted the Greek alphabet and adapted Greek myth.  The Etruscan language is non-Indo-European.
\end{document}
